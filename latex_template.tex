%%%%%%%%%%%5 το δοκίμασα κια δουλευει στον editor winshell %%%%%%
%%%%%%%%%%%%%%%%%%%%%%%%%%%%%%%%%%%%%%%%%%%%
\documentclass[12pt]{amsart}

\usepackage{amsfonts}
\usepackage{amscd}
\usepackage{amssymb}
\usepackage[english,greek]{babel}
\usepackage[iso-8859-7]{inputenc}
\usepackage[algosection,lined,boxed,commentsnumbered,ruled,vlined]{algorithm2e}
\NoCaptionOfAlgo
\def\tl{\textlatin }
\newtheorem{algor}{\bf{Αλγόριθμος}}[subsection]

\begin{document}

Έδω γράφουμε το κείμενο
Οι μαθηματικες σχέσεις ως εξής : $x^2+1=0 $
ή κεντραρισμένες $$x^2+1=0$$
ή
\begin{equation}
x^2+1=0
\end{equation}

Για να γράψω κώδικα π.χ.

\begin{algorithm}[H]
\footnotesize {
\begin{algor} : {Κατασκευή μετάθεσης $S$} \label{rc4-1} \
{\bf Είσοδος.} \tl{seed} (40 με 256 \tl{bits})\
{\bf Έξοδος.} αρχική τιμή της μετάθεσης $S$
\end{algor}
\SetAlgoLined
  \tl{
\nl seedlen = length(seed)\
\nl \For {$i=0$ to $255$}{
\nl  $S[i]=i$
                        }
\nl $j=0$\
\nl \For {$i=0$ to $255$}{ 
\nl $j=(j+S[i]+seed[i\ \mod seedlen])\mod 256$\
\nl $S[i]\leftrightarrow S[j]$       
                }}}
\end{algorithm}

η βιβλιογραφία ως εξής, π.χ.

\begin{thebibliography}{9}

\bibitem{HCohen} \tl{ Cohen, Henri; Advanced topics in computational number
theory. Graduate Texts in Mathematics, 193. Springer-Verlag, New York, 2000.}
\end{thebibliography}



\end{document}
